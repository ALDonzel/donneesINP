% Options for packages loaded elsewhere
\PassOptionsToPackage{unicode}{hyperref}
\PassOptionsToPackage{hyphens}{url}
%
\documentclass[
]{article}
\usepackage{lmodern}
\usepackage{amssymb,amsmath}
\usepackage{ifxetex,ifluatex}
\ifnum 0\ifxetex 1\fi\ifluatex 1\fi=0 % if pdftex
  \usepackage[T1]{fontenc}
  \usepackage[utf8]{inputenc}
  \usepackage{textcomp} % provide euro and other symbols
\else % if luatex or xetex
  \usepackage{unicode-math}
  \defaultfontfeatures{Scale=MatchLowercase}
  \defaultfontfeatures[\rmfamily]{Ligatures=TeX,Scale=1}
\fi
% Use upquote if available, for straight quotes in verbatim environments
\IfFileExists{upquote.sty}{\usepackage{upquote}}{}
\IfFileExists{microtype.sty}{% use microtype if available
  \usepackage[]{microtype}
  \UseMicrotypeSet[protrusion]{basicmath} % disable protrusion for tt fonts
}{}
\makeatletter
\@ifundefined{KOMAClassName}{% if non-KOMA class
  \IfFileExists{parskip.sty}{%
    \usepackage{parskip}
  }{% else
    \setlength{\parindent}{0pt}
    \setlength{\parskip}{6pt plus 2pt minus 1pt}}
}{% if KOMA class
  \KOMAoptions{parskip=half}}
\makeatother
\usepackage{xcolor}
\IfFileExists{xurl.sty}{\usepackage{xurl}}{} % add URL line breaks if available
\IfFileExists{bookmark.sty}{\usepackage{bookmark}}{\usepackage{hyperref}}
\hypersetup{
  pdftitle={R Notebook},
  hidelinks,
  pdfcreator={LaTeX via pandoc}}
\urlstyle{same} % disable monospaced font for URLs
\usepackage[margin=1in]{geometry}
\usepackage{color}
\usepackage{fancyvrb}
\newcommand{\VerbBar}{|}
\newcommand{\VERB}{\Verb[commandchars=\\\{\}]}
\DefineVerbatimEnvironment{Highlighting}{Verbatim}{commandchars=\\\{\}}
% Add ',fontsize=\small' for more characters per line
\usepackage{framed}
\definecolor{shadecolor}{RGB}{248,248,248}
\newenvironment{Shaded}{\begin{snugshade}}{\end{snugshade}}
\newcommand{\AlertTok}[1]{\textcolor[rgb]{0.94,0.16,0.16}{#1}}
\newcommand{\AnnotationTok}[1]{\textcolor[rgb]{0.56,0.35,0.01}{\textbf{\textit{#1}}}}
\newcommand{\AttributeTok}[1]{\textcolor[rgb]{0.77,0.63,0.00}{#1}}
\newcommand{\BaseNTok}[1]{\textcolor[rgb]{0.00,0.00,0.81}{#1}}
\newcommand{\BuiltInTok}[1]{#1}
\newcommand{\CharTok}[1]{\textcolor[rgb]{0.31,0.60,0.02}{#1}}
\newcommand{\CommentTok}[1]{\textcolor[rgb]{0.56,0.35,0.01}{\textit{#1}}}
\newcommand{\CommentVarTok}[1]{\textcolor[rgb]{0.56,0.35,0.01}{\textbf{\textit{#1}}}}
\newcommand{\ConstantTok}[1]{\textcolor[rgb]{0.00,0.00,0.00}{#1}}
\newcommand{\ControlFlowTok}[1]{\textcolor[rgb]{0.13,0.29,0.53}{\textbf{#1}}}
\newcommand{\DataTypeTok}[1]{\textcolor[rgb]{0.13,0.29,0.53}{#1}}
\newcommand{\DecValTok}[1]{\textcolor[rgb]{0.00,0.00,0.81}{#1}}
\newcommand{\DocumentationTok}[1]{\textcolor[rgb]{0.56,0.35,0.01}{\textbf{\textit{#1}}}}
\newcommand{\ErrorTok}[1]{\textcolor[rgb]{0.64,0.00,0.00}{\textbf{#1}}}
\newcommand{\ExtensionTok}[1]{#1}
\newcommand{\FloatTok}[1]{\textcolor[rgb]{0.00,0.00,0.81}{#1}}
\newcommand{\FunctionTok}[1]{\textcolor[rgb]{0.00,0.00,0.00}{#1}}
\newcommand{\ImportTok}[1]{#1}
\newcommand{\InformationTok}[1]{\textcolor[rgb]{0.56,0.35,0.01}{\textbf{\textit{#1}}}}
\newcommand{\KeywordTok}[1]{\textcolor[rgb]{0.13,0.29,0.53}{\textbf{#1}}}
\newcommand{\NormalTok}[1]{#1}
\newcommand{\OperatorTok}[1]{\textcolor[rgb]{0.81,0.36,0.00}{\textbf{#1}}}
\newcommand{\OtherTok}[1]{\textcolor[rgb]{0.56,0.35,0.01}{#1}}
\newcommand{\PreprocessorTok}[1]{\textcolor[rgb]{0.56,0.35,0.01}{\textit{#1}}}
\newcommand{\RegionMarkerTok}[1]{#1}
\newcommand{\SpecialCharTok}[1]{\textcolor[rgb]{0.00,0.00,0.00}{#1}}
\newcommand{\SpecialStringTok}[1]{\textcolor[rgb]{0.31,0.60,0.02}{#1}}
\newcommand{\StringTok}[1]{\textcolor[rgb]{0.31,0.60,0.02}{#1}}
\newcommand{\VariableTok}[1]{\textcolor[rgb]{0.00,0.00,0.00}{#1}}
\newcommand{\VerbatimStringTok}[1]{\textcolor[rgb]{0.31,0.60,0.02}{#1}}
\newcommand{\WarningTok}[1]{\textcolor[rgb]{0.56,0.35,0.01}{\textbf{\textit{#1}}}}
\usepackage{graphicx}
\makeatletter
\def\maxwidth{\ifdim\Gin@nat@width>\linewidth\linewidth\else\Gin@nat@width\fi}
\def\maxheight{\ifdim\Gin@nat@height>\textheight\textheight\else\Gin@nat@height\fi}
\makeatother
% Scale images if necessary, so that they will not overflow the page
% margins by default, and it is still possible to overwrite the defaults
% using explicit options in \includegraphics[width, height, ...]{}
\setkeys{Gin}{width=\maxwidth,height=\maxheight,keepaspectratio}
% Set default figure placement to htbp
\makeatletter
\def\fps@figure{htbp}
\makeatother
\setlength{\emergencystretch}{3em} % prevent overfull lines
\providecommand{\tightlist}{%
  \setlength{\itemsep}{0pt}\setlength{\parskip}{0pt}}
\setcounter{secnumdepth}{-\maxdimen} % remove section numbering

\title{R Notebook}
\author{}
\date{\vspace{-2.5em}}

\begin{document}
\maketitle

\begin{Shaded}
\begin{Highlighting}[]
\KeywordTok{library}\NormalTok{(tidyverse)}
\end{Highlighting}
\end{Shaded}

\begin{verbatim}
## -- Attaching packages --------------------------------------- tidyverse 1.3.1 --
\end{verbatim}

\begin{verbatim}
## v ggplot2 3.3.5     v purrr   0.3.4
## v tibble  3.1.5     v dplyr   1.0.7
## v tidyr   1.1.4     v stringr 1.4.0
## v readr   2.0.2     v forcats 0.5.1
\end{verbatim}

\begin{verbatim}
## -- Conflicts ------------------------------------------ tidyverse_conflicts() --
## x dplyr::filter() masks stats::filter()
## x dplyr::lag()    masks stats::lag()
\end{verbatim}

\begin{Shaded}
\begin{Highlighting}[]
\KeywordTok{library}\NormalTok{(ggplot2)}
\KeywordTok{library}\NormalTok{(readxl)}
\end{Highlighting}
\end{Shaded}

\hypertarget{nombre-de-personne-par-spuxe9cialituxe9s}{%
\section{Nombre de personne par
spécialités}\label{nombre-de-personne-par-spuxe9cialituxe9s}}

\begin{Shaded}
\begin{Highlighting}[]
\NormalTok{data \textless{}{-}}\StringTok{ }\KeywordTok{read\_excel}\NormalTok{(}\DataTypeTok{path =} \StringTok{"annuaireINP\_complet.xls"}\NormalTok{)}
\NormalTok{data }\OperatorTok{\%\textgreater{}\%}
\StringTok{  }\KeywordTok{count}\NormalTok{(Spécialité, }\DataTypeTok{name=} \StringTok{"Nombre"}\NormalTok{)}
\end{Highlighting}
\end{Shaded}

\begin{verbatim}
## # A tibble: 6 x 2
##   Spécialité     Nombre
##   <chr>           <int>
## 1 Archéologie       137
## 2 Archives          283
## 3 Autre carrière     13
## 4 MHI               180
## 5 Musées            472
## 6 PSTN               43
\end{verbatim}

\hypertarget{nombre-de-femmes-et-dhommes-quelque-soit-la-spuxe9cialituxe9}{%
\section{Nombre de femmes et d'hommes quelque soit la
spécialité}\label{nombre-de-femmes-et-dhommes-quelque-soit-la-spuxe9cialituxe9}}

\begin{Shaded}
\begin{Highlighting}[]
\NormalTok{data \textless{}{-}}\StringTok{ }\KeywordTok{read\_excel}\NormalTok{(}\DataTypeTok{path =} \StringTok{"annuaireINP\_complet.xls"}\NormalTok{)}
\NormalTok{data }\OperatorTok{\%\textgreater{}\%}\StringTok{ }
\StringTok{  }\KeywordTok{count}\NormalTok{ (Sexe, }\DataTypeTok{name=} \StringTok{"Nombre"}\NormalTok{)}
\end{Highlighting}
\end{Shaded}

\begin{verbatim}
## # A tibble: 2 x 2
##   Sexe  Nombre
##   <chr>  <int>
## 1 F        691
## 2 H        437
\end{verbatim}

\hypertarget{ruxe9partition-par-spuxe9cialituxe9}{%
\section{Répartition par
spécialité}\label{ruxe9partition-par-spuxe9cialituxe9}}

\hypertarget{archuxe9ologie}{%
\subsection{Archéologie}\label{archuxe9ologie}}

\begin{Shaded}
\begin{Highlighting}[]
\NormalTok{data \textless{}{-}}\StringTok{ }\KeywordTok{read\_excel}\NormalTok{(}\DataTypeTok{path =} \StringTok{"annuaireINP\_complet.xls"}\NormalTok{)}
\NormalTok{data }\OperatorTok{\%\textgreater{}\%}
\StringTok{  }\KeywordTok{filter}\NormalTok{(Spécialité }\OperatorTok{==}\StringTok{ "Archéologie"}\NormalTok{) }\OperatorTok{\%\textgreater{}\%}\StringTok{ }
\StringTok{  }\KeywordTok{count}\NormalTok{(Sexe, }\DataTypeTok{name =} \StringTok{"Nombre"}\NormalTok{)}
\end{Highlighting}
\end{Shaded}

\begin{verbatim}
## # A tibble: 2 x 2
##   Sexe  Nombre
##   <chr>  <int>
## 1 F         70
## 2 H         67
\end{verbatim}

\hypertarget{archives}{%
\subsection{Archives}\label{archives}}

\begin{Shaded}
\begin{Highlighting}[]
\NormalTok{data \textless{}{-}}\StringTok{ }\KeywordTok{read\_excel}\NormalTok{(}\DataTypeTok{path =} \StringTok{"annuaireINP\_complet.xls"}\NormalTok{)}
\NormalTok{data }\OperatorTok{\%\textgreater{}\%}
\StringTok{  }\KeywordTok{filter}\NormalTok{(Spécialité }\OperatorTok{==}\StringTok{ "Archives"}\NormalTok{) }\OperatorTok{\%\textgreater{}\%}\StringTok{ }
\StringTok{  }\KeywordTok{count}\NormalTok{(Sexe, }\DataTypeTok{name =} \StringTok{"Nombre"}\NormalTok{)}
\end{Highlighting}
\end{Shaded}

\begin{verbatim}
## # A tibble: 2 x 2
##   Sexe  Nombre
##   <chr>  <int>
## 1 F        160
## 2 H        123
\end{verbatim}

\begin{Shaded}
\begin{Highlighting}[]
\NormalTok{data \textless{}{-}}\StringTok{ }\KeywordTok{read\_excel}\NormalTok{(}\DataTypeTok{path =} \StringTok{"annuaireINP\_complet.xls"}\NormalTok{)}
\NormalTok{data }\OperatorTok{\%\textgreater{}\%}
\StringTok{  }\KeywordTok{filter}\NormalTok{(Spécialité }\OperatorTok{==}\StringTok{ "Autre carrière"}\NormalTok{) }\OperatorTok{\%\textgreater{}\%}\StringTok{ }
\StringTok{  }\KeywordTok{count}\NormalTok{(Sexe, }\DataTypeTok{name =} \StringTok{"Nombre"}\NormalTok{)}
\end{Highlighting}
\end{Shaded}

\begin{verbatim}
## # A tibble: 2 x 2
##   Sexe  Nombre
##   <chr>  <int>
## 1 F          8
## 2 H          5
\end{verbatim}

\hypertarget{mhi}{%
\subsection{MHI}\label{mhi}}

\begin{Shaded}
\begin{Highlighting}[]
\NormalTok{data \textless{}{-}}\StringTok{ }\KeywordTok{read\_excel}\NormalTok{(}\DataTypeTok{path =} \StringTok{"annuaireINP\_complet.xls"}\NormalTok{)}
\NormalTok{data }\OperatorTok{\%\textgreater{}\%}
\StringTok{  }\KeywordTok{filter}\NormalTok{(Spécialité }\OperatorTok{==}\StringTok{ "MHI"}\NormalTok{) }\OperatorTok{\%\textgreater{}\%}\StringTok{ }
\StringTok{  }\KeywordTok{count}\NormalTok{(Sexe, }\DataTypeTok{name =} \StringTok{"Nombre"}\NormalTok{)}
\end{Highlighting}
\end{Shaded}

\begin{verbatim}
## # A tibble: 2 x 2
##   Sexe  Nombre
##   <chr>  <int>
## 1 F        116
## 2 H         64
\end{verbatim}

\hypertarget{musuxe9es}{%
\subsection{Musées}\label{musuxe9es}}

\begin{Shaded}
\begin{Highlighting}[]
\NormalTok{data \textless{}{-}}\StringTok{ }\KeywordTok{read\_excel}\NormalTok{(}\DataTypeTok{path =} \StringTok{"annuaireINP\_complet.xls"}\NormalTok{)}
\NormalTok{data }\OperatorTok{\%\textgreater{}\%}
\StringTok{  }\KeywordTok{filter}\NormalTok{(Spécialité }\OperatorTok{==}\StringTok{ "Musées"}\NormalTok{) }\OperatorTok{\%\textgreater{}\%}\StringTok{ }
\StringTok{  }\KeywordTok{count}\NormalTok{(Sexe, }\DataTypeTok{name =} \StringTok{"Nombre"}\NormalTok{)}
\end{Highlighting}
\end{Shaded}

\begin{verbatim}
## # A tibble: 2 x 2
##   Sexe  Nombre
##   <chr>  <int>
## 1 F        311
## 2 H        161
\end{verbatim}

\hypertarget{pstn}{%
\subsection{PSTN}\label{pstn}}

\begin{Shaded}
\begin{Highlighting}[]
\NormalTok{data \textless{}{-}}\StringTok{ }\KeywordTok{read\_excel}\NormalTok{(}\DataTypeTok{path =} \StringTok{"annuaireINP\_complet.xls"}\NormalTok{)}
\NormalTok{data }\OperatorTok{\%\textgreater{}\%}
\StringTok{  }\KeywordTok{filter}\NormalTok{(Spécialité }\OperatorTok{==}\StringTok{ "PSTN"}\NormalTok{) }\OperatorTok{\%\textgreater{}\%}\StringTok{ }
\StringTok{  }\KeywordTok{count}\NormalTok{(Sexe, }\DataTypeTok{name =} \StringTok{"Nombre"}\NormalTok{)}
\end{Highlighting}
\end{Shaded}

\begin{verbatim}
## # A tibble: 2 x 2
##   Sexe  Nombre
##   <chr>  <int>
## 1 F         26
## 2 H         17
\end{verbatim}

\end{document}
